%----------------------------------------------------------------------------
\documentclass[a4paper]{report}
\usepackage{times}

%----------------------------------------------------------------------------
\input{../../../util/tex/style}
\input{../../../util/tex/products}
\input{../../../util/doc/cmdline/options}
\def\version{3.1}


%----------------------------------------------------------------------------
\def\setupjar{jcs-setup.jar}
\def\javadoc{{\jcs} API documentation}
\def\directory{jcs}
\def\product{\jcs}

%----------------------------------------------------------------------------
\begin{document}
\title{{\jcs} (v\version) tutorial}
\author{Copyright 2002-2006 \koalog}
\date{}
\maketitle

\begin{center}
\begin{minipage}{9cm}
\begin{small}
{\sunlegal}\\

{\apachelegal}\\
\end{small}
\end{minipage}
\end{center}

\tableofcontents

%----------------------------------------------------------------------------
\chapter{About this guide}
This chapter gives various information about this guide.

%----------------------------------------------------------------------------
\section{Who should use this guide}
This guide is intended for programmers who develop applications using {\jcs}. 
In order to benefit from this guide, you should have a basic knowledge of the 
{\java} programming language. No experience with any other {\koalog} 
product is required.
 
%----------------------------------------------------------------------------
\section{Contents of this guide}
This guide is organized as follows:
\begin{itemize}
\item Chapter \ref{chap:introduction} 
introduces basic concepts about combinatorial problems and constraint programming. 
Then, it explains a couple of technical choices concerning {\jcs}.
\item Chapter \ref{chap:using} explains how to install and use {\jcs}.
It also contains a very short tutorial based on the study of a simple example.
\item Chapter \ref{chap:domains-variables} introduces the concepts of domain and variable.
\item Chapter \ref{chap:constraints-problems} presents the concept of
  constraint and explain how constraints can be grouped into problems.
\item Chapter \ref{chap:solvers-heuristics} goes into the details of solvers and 
explain how they use heuristics to solve the problems.
\item Chapter \ref{chap:optimizers} introduces the generic concept of optimizer.
\end{itemize}

%----------------------------------------------------------------------------
\section{How to use this guide}
If you are already familiar with constraint programming, then you can directly jump 
to chapter \ref{chap:using}, otherwise the reading of 
chapter \ref{chap:introduction} is necessary. 
In both cases, for a first reading, it is recommended to read the chapters 
in order.

%----------------------------------------------------------------------------
\section{Conventions}
{\java} code examples will be displayed as follows:
\begin{verbatim}
public class HelloWorld {
   public static void main(String[] args) {
      System.out.println("Hello world");
   }
}
\end{verbatim}

%----------------------------------------------------------------------------
\section{Related documents}
Other information can be found in:
\begin{itemize}
\item ``An overview of {\jcs}'',
\item ``The {\javadoc}''.
\end{itemize}

%----------------------------------------------------------------------------
\chapter{Introduction}
\label{chap:introduction}
This chapter introduces basic concepts about combinatorial problems 
and constraint programming. 
Then, it explains a couple of technical choices concerning {\jcs}.

%----------------------------------------------------------------------------
\section{Combinatorial problems}
A combinatorial problem is defined by a set of 
finite domain\footnote{Usually these variables are integers and are bound 
by a minimal and a maximal value.}  
variables and 
a set of constraints (or relations) over these variables.

% MAT: ajout de paragraphe
Thus, solving such a problem means finding a value for each of the problem
variables, so that all problem constraints are satisfied. But finding a
solution to the problem is not always enough, and it is for instance often
important to find the best solution, given that the possible solutions to the
problem can be compared to each other.

We can therefore distinguish two categories of combinatorial problems:
\begin{itemize}
\item satisfaction problems;
\item optimization problems.
\end{itemize}
In both cases, a solution will always correspond to an instantiation of the
problem variables.

\subsection{Satisfaction problems}
A satisfaction problem is a problem where one searches for:
\begin{itemize}
\item a solution\footnote{In some cases, the set of all solutions.} or 
\item a proof that there is no solution.
\end{itemize}
Real life examples include, among others:
\begin{itemize}
\item resource allocation;
\item time-tabling;
\item configuration (when one tries to find a feasible configuration).
\end{itemize}

\subsection{Optimization problems}
An optimization problem is a problem where one searches for an optimal
solution. Optimal means here that one of the problem variables takes an
optimal, minimal or maximal, value.

Real life examples include, among others:
\begin{itemize}
\item scheduling (when one tries to minimize the total duration of a project),
\item configuration (when one tries to find a feasible configuration of
  minimal price).
\end{itemize}

%----------------------------------------------------------------------------
\section{Constraint Programming}
The technique of Constraint Programming was introduced in the late eighties.

\subsection{Overview}
Instead of writing complex, specific and error-prone algorithms 
in a low-level language, the idea is to use a high-level language that will
provide, at least, two layers:
\begin{itemize}
\item a modelling layer, facilitating the description of the problem;
\item a solver or control layer, responsible for the resolution of the problem.
\end{itemize}

\subsection{The three layers of {\jcs}}
In addition to the modelling and control layers, {\jcs} offers a third layer,
the algorithms layer.
Both the solver and the algorithms layers can easily be customized:
\begin{itemize}
\item customization of the solver layer (see chapter \ref{chap:solvers-heuristics}): 
the idea is to control the way the solver searches for a solution, 
this customization can consist in changing the heuristics used by the solver,
\item customization of the algorithm layer (see chapter \ref{chap:constraints-problems}):
the idea is to change the filtering algorithms used in the constraints, 
this customization can consist in designing a new constraint. 
\end{itemize}
In both cases, these customizations can change:
\begin{itemize}
\item the order in which the solutions of the problem are found 
(this is easy to understand for the solver layer, 
this is also the case for the algorithm layer because 
filtering algorithm affect the way the domains are refined and 
heuristics may depend on the size of the domains),
\item the time needed for finding them,
\item in the case of optimization, the optimal solution 
(if there is more than one optimal solution).
\end{itemize}
But they will not change:
\begin{itemize}
\item the set of all solutions,
\item in the case of optimization, the optimal value.
\end{itemize}
 
%----------------------------------------------------------------------------
\section{What is {\jcs}?}
{\jcs} is a {\java} library for Constraint Programming. 
The choice of a library rather than a language (such as Prolog) 
makes the integration of {\jcs} programs into other applications seamless.

The choice of {\java} programming language makes {\jcs} platform-independent.
{\jcs} also benefits from well-known standards in the growing 
{\java} community:
\begin{itemize}
\item {\log4j} for its logging system;
\item {\junit} for its unit tests.      
\end{itemize}

%----------------------------------------------------------------------------
\chapter{Using {\jcs}}
\label{chap:using}
This chapter explains how to install and use {\jcs}. 
It also contains a very short tutorial based on the study of a simple example.

%----------------------------------------------------------------------------
\section{Installation and usage}
This section explains how to install and use {\jcs}.

\subsection{Installing {\jcs}}
\label{subsec:install}
\input{../../../installer/doc/technical/userguide}
  
\subsection{Using {\jcs}}
This sub-section explains how to use {\jcs}.

\subsubsection{License}
\input{../../../license/doc/technical/userguide}

The installation has created a {\tt bin} directory 
which contains a script {\tt kcs} that you can use to test the version
of {\jcs} you are running.

\helpusage{kcs}

\versionusage{\jcs}{kcs}

\subsubsection{Classpath}
The installation has created a {\tt lib} directory which contains:
\begin{itemize}
\item the {\jcs} jar file,
\item additional jar files corresponding to {\java} librairies used by {\jcs}.
\end{itemize}

To compile and run any {\java} program using {\jcs}, 
do not forget to include all the jar files, contained in directory {\tt lib}, 
in your classpath.

\subsection{{\log4j} configuration}
{\jcs} uses {\log4j} for its logging system.
{\log4j} can be configured\footnote{It also offers a default configuration 
but we do not recommended it since you surely want to keep control of what is logged by your program.}
from a properties file or from an XML file.

As an example, 
you can configure {\log4j} with the properties file {\tt log4j.properties} given in figure \ref{tab:code-log4j}, 
by:
\begin{verbatim}
import org.apache.log4j.PropertyConfigurator;

public class A {
   public static void main(String[] args) {
        PropertyConfigurator.configure("log4j.properties");
        % start using Koalog Constraint Solver below
   }
}
\end{verbatim}
\begin{figure}[htb]
\begin{centeredframe}
\begin{verbatim}
log4j.rootCategory=INFO, A1

log4j.appender.A1=org.apache.log4j.ConsoleAppender
log4j.appender.A1.layout=org.apache.log4j.PatternLayout
log4j.appender.A1.layout.ConversionPattern=%m%n
\end{verbatim}
\end{centeredframe}
\caption{A simple {\log4j} configuration file\label{tab:code-log4j}}
\end{figure}

Refer to {\log4j} documentation for further details.
 
%----------------------------------------------------------------------------
\section{Quick tutorial}
\label{sec:send}
In this section, we present the complete study of 
the famous crypto-arithmetic puzzle $SEND+MORE=MONEY$.
The problem consists in finding, for each of the letters $\{S,E,N,D,M,O,R,Y\}$ a value between $0$ and $9$ such that:
\begin{itemize}
\item $SEND$, $MORE$ and $MONEY$ represent numbers, and thus should not start
  with a $0$;
\item $S$,$E$,$N$,$D$,$M$,$O$,$R$,$Y$ are all different,
\item the sum of the numbers $SEND$ and $MORE$ is $MONEY$.
\end{itemize}

You probably have already noticed that this problem is a satisfaction problem.
Let us go a bit further and demonstrate that this problem, even if it can be
described shortly, is not straightforward. Each variable can take ten values,
from $0$ to $9$, and there are $8$ variables. The search space size is
therefore $10^8$. A custom algorithm would probably try to enumerate the
100\ 000\ 000 instantiations in order to find the unique solution.

Let us explain how we can easily solve the problem using {\jcs}.
The first thing to do is to declare and define a problem:
\begin{verbatim}
Problem pb = new BaseProblem();
\end{verbatim}
Then one has to declare and define the problem variables:
\begin{verbatim}
IntegerVariable s = new IntegerVariable("s",1,9);
IntegerVariable e = new IntegerVariable("e",0,9);
IntegerVariable n = new IntegerVariable("n",0,9);
IntegerVariable d = new IntegerVariable("d",0,9);
IntegerVariable m = new IntegerVariable("m",1,9);
IntegerVariable o = new IntegerVariable("o",0,9);
IntegerVariable r = new IntegerVariable("r",0,9);
IntegerVariable y = new IntegerVariable("y",0,9);
\end{verbatim}
Let us consider the construction of variable {\tt s} with
{\tt new IntegerVariable("s",1,9);}. The parameter {\tt "s"} is the {\jcs}
name for the
{\java} variable {{\tt s} and will be used by the solver for pretty-printing. 
The parameters {\tt 1} and {\tt 9} define the 
range\footnote{We know that $S \neq 0$ because it starts the word $SEND$, 
for the same reason $M \neq 0$.}
of values for the variable {\tt s}.  
Then, we have to assign these variables to the problem:
\begin{verbatim}
IntegerVariable[] vars = {s,e,n,d,m,o,r,y};
pb.setVariables(vars);
\end{verbatim}
The problem variables array will be used by the solver:
\begin{itemize}
\item to search for solutions;
\item to display the solutions found.
\end{itemize}
Now, we have to define the constraints on the variables.
We know that the variables have to be different:
\begin{verbatim}
pb.add(new AllDifferent(vars));
\end{verbatim}
Let us go back to the equation $SEND+MORE=MONEY$ before defining the last
constraint.
This equation translates into 
$1000S + 100E + 10N + D + 1000M +100O + 10R + E = 10000M +1000O +100N +10E + Y$, 
or equivalently $1000S + 91E - 90N + D - 9000M - 900O + 10R -Y = 0$.
This can be expressed in {\jcs} by:
\begin{verbatim}
pb.add(new ConstantWeightedSum(new int[] {1000,
                                          91,
                                          -90,
                                          1,
                                          -9000,
                                          -900,
                                          10,
                                          -1}, 
                               vars,
                               0));
\end{verbatim}
We have now completely modeled the problem using the first layer of {\jcs}. 
Let us use the second layer to declare and define a solver:
\begin{verbatim}
Solver solver = new DefaultSplitSolver(pb);
\end{verbatim}
As we said before, there is only one solution, 
thus we are going to search for the two first solutions and see what happens:
\begin{verbatim}
solver.solve(2);
\end{verbatim}
The complete source code for the problem $SEND+MORE=MONEY$ is rather short and
is given in figure \ref{tab:code-send}.
\begin{figure}[htb]
\begin{centeredframe}
\begin{verbatim}
Problem pb = new BaseProblem();
IntegerVariable s = new IntegerVariable("s",1,9);
IntegerVariable e = new IntegerVariable("e",0,9);
IntegerVariable n = new IntegerVariable("n",0,9);
IntegerVariable d = new IntegerVariable("d",0,9);
IntegerVariable m = new IntegerVariable("m",1,9);
IntegerVariable o = new IntegerVariable("o",0,9);
IntegerVariable r = new IntegerVariable("r",0,9);
IntegerVariable y = new IntegerVariable("y",0,9);
IntegerVariable[] vars = {s,e,n,d,m,o,r,y};
pb.setVariables(vars);
pb.add(new AllDifferent(vars));
pb.add(new ConstantWeightedSum(new int[] {1000,
                                          91,
                                          -90,
                                          1,
                                          -9000,
                                          -900,
                                          10,
                                          -1}, 
                     vars,
                     0));
Solver solver = new DefaultSplitSolver(pb);
solver.solve(2);
\end{verbatim}
\end{centeredframe}
\caption{Problem $SEND+MORE=MONEY$ source code\label{tab:code-send}}
\end{figure}

The output\footnote{It can be fully customized.} of {\jcs} is:
\begin{verbatim}
solution found in 3 ms
1 backtrack(s), 51 filter(s), max depth 1
no more solution
solved in 7 ms
s e n d m o r y
9 5 6 7 1 0 8 2
3 backtracks in total, 69 filters in total, max depth 1
\end{verbatim}
This means:
\begin{itemize}
\item the first solution was found in $3 ms$ and 1 backtrack, 51 constraints were filtered, the maximal depth in the search tree was 1;
\item no other solution can be found;
\item the proof of this took $7 ms$ and 3 backtracks, 69 constraints were filtered, the maximal depth in the search tree was 1;
\item the unique solution is:
%$$\{D=7 \wedge E=5 \wedge M=1 \wedge N=6 \wedge O=0 \wedge R=8 \wedge S=9
%\wedge Y=2\}$$
%MAT: je trouve plus lisible:
$$\{D=7, E=5, M=1, N=6, O=0, R=8, S=9, Y=2\}$$
\end{itemize}

%----------------------------------------------------------------------------
\chapter{Domains and variables}
\label{chap:domains-variables}
This chapter introduces the concepts of domains and 
variables.

%----------------------------------------------------------------------------
\section{Domains}
A domain defines the set of possible values for a variable 
(respectively represented in {\jcs} by the {\tt Domain} and the {\tt Variable} interfaces).

Since we are interested here in solving combinatorial problems, we will focus, in the rest
of this tutorial, on integer domains and integer variables (these are respectively
represented in {\jcs} by 
the {\tt IntegerDomain} and the {\tt IntegerVariable} classes). 

Note that {\jcs} can also handle set domains and set variables (these are respectively
represented in {\jcs} by 
the {\tt SetDomain} and the {\tt SetVariable} classes) but it is out of the scope of this document.
Refer to the {\javadoc} for more informations on set domains, set variables and set constraints.

During the search for a solution by a solver, each integer domain is refined
until the corresponding variable is instantiated.

\subsection{Intervals}
You can declare and define a domain by:
\begin{verbatim}
IntegerDomain d = new MinMaxDomain(1, 5);
\end{verbatim}
This domain contains the integers from $1$ to $5$. 
It is represented internally as an interval 
which minimal value is $1$ and which maximal value is $5$.

\subsection{Sparse domains}
Alternatively, you could write:
\begin{verbatim}
IntegerDomain d = new SparseDomain(1, 5);
\end{verbatim}
The set of possible values is the same, but the internal representation
differs: in this case, it is a set containing the elements $\{ 1, 2, 3, 4, 5
\}$. We will see in chapter \ref{chap:constraints-problems}
how the algorithm layer can, in some cases, take advantage of this second definition.

Refer to the {\javadoc} for other ways of creating domains.

%----------------------------------------------------------------------------
\section{Variables}
A variable has a name and a domain\footnote{This domain is likely to change
  during the search for a solution, so domains should not
  be shared among variables.}.
You can declare and define a variable by:
\begin{verbatim}
Variable v = new IntegerVariable("a name", domain);
\end{verbatim}
In section \ref{sec:send}, we have already seen an interesting shortcut
of the form:
\begin{verbatim}
Variable v = new IntegerVariable("a name", 1, 5);
\end{verbatim}
This is strictly equivalent to:
\begin{verbatim}
Variable v = new IntegerVariable("a name", 
                                 new MinMaxDomain(1, 5));
\end{verbatim}
Refer to the {\javadoc} for other ways of creating variables.

%----------------------------------------------------------------------------
\chapter{Constraints and problems}
\label{chap:constraints-problems}
This chapter presents the concept of constraints and 
explains how constraints can be grouped into problems.

%----------------------------------------------------------------------------
\section{Constraints}
Constraints are one of the pillars of {\jcs}.
They correspond to mathematical relations over variables,
and encapsulate filtering algorithms (the algorithm layer)
that are applied during the search for a solution.

\subsection{Pre-defined constraints}
{\jcs} comes with a large collection of constraints such as:
\begin{itemize}
\item arithmetic constraints ($x=y+z$, $x \leq y$, $x=|y|$, etc),
\item boolean constraints ($x=y \wedge z$, $x = \neg y$, etc),
\item meta-constraints ($b <=> (x=c)$, $b <=> (x \leq y)$, etc),
\item global constraints ($x = [y_1, \ldots, y_n]_i$, $\wedge_{i \neq j} x_i \neq x_j$, etc).
\end{itemize}
Refer to the {\javadoc} for the complete list of pre-defined constraints.

\subsection{A constraint from the inside}
{\jcs} contains a lot of pre-defined constraints and 
their corresponding filtering algorithms (the algorithm layer).
However, one may wish to write his own constraints:
\begin{itemize}
\item either because a needed constraint is not provided by {\jcs},
\item or because a new filtering algorithm has been discovered\footnote{This
    is, for example, the case for the constraint of {\tt AllDifferent}
    (corresponding to the relation $\wedge_{i \neq j} x_i \neq x_j$) for which
    the search of new filtering algorithms is an active field of research.}
  for an existing constraint.
\end{itemize}

Let us consider the relation $x \leq y$ and write the 
corresponding constraint {\tt Leq} (less or equal). This relation is 
binary\footnote{Note that {\jcs} comes with the following pre-defined classes:
\begin{itemize}
\item {\tt UnaryConstraint} (constraints constraining a single variable),
\item {\tt BinaryConstraint} (constraints constraining two variables),
\item {\tt TernaryConstraint} (constraints constraining three variables),
\item {\tt BaseConstraint} (constraints constraining $n$ variables).
\end{itemize}}, which means that it involves two variables:
\begin{verbatim}  
public class Leq extends BinaryConstraint {
\end{verbatim}
Let us give a name to this constraint:
\begin{verbatim} 
   public Leq(IntegerVariable x, IntegerVariable y) {
      super(x, y);
      name = "x<=y";
   }
\end{verbatim}
There are only two methods that have to be implemented: 
\begin{itemize}
\item the {\tt filter} method,
\item the {\tt updateDependencies} method.
\end{itemize}

\subsubsection{Method {\tt filter}}
We can access the constraint variables by:
\begin{verbatim}
      IntegerVariable x = (IntegerVariable) variables[0];
      IntegerVariable y = (IntegerVariable) variables[1];
\end{verbatim}
Now comes the filtering algorithm, 
which generally consists in a sequence of min and max adjustments.

The {\tt IntegerVariable} interface provides two main methods, {\tt adjustMin} and
{\tt adjustMax}, that are responsible for applying these adjustments. Note that
simple accessors such as {\tt setMin} and {\tt setMax} should not be used:
{\jcs} solvers require any modification of the variables domains to be
backtrackable, or, in other terms, undoable. Thus, {\tt adjustMin} and
{\tt adjustMax} not only modify the domain boundaries, as {\tt setMin} and
{\tt setMax} would, but also record the modification so that a {\jcs} solver
can later undo it. That is what the {\tt ChoicePointStack} class is used for,
but its description is out of the scope of this document.

In the case of the {\tt Leq} constraint, the filtering algorithm reduces to:
\begin{verbatim}
      x.adjustMax(cp, cs, this, y.getMax());
      y.adjustMin(cp, cs, this, x.getMin());
\end{verbatim}
It means that:
\begin{itemize}
\item the new maximal value of $x$ is less or equal than the previous maximal
  value of $y$ and
\item the new minimal value of $y$ is greater or equal than the previous
  minimal value of $x$.
% MAT: v�rifie bien que je ne me suis pas tromp�. Avant, il y avait:
% \item the new minimal value of $y$ is less or equal than the previous
% minimal value of $x$.
\end{itemize}
Refer to the {\javadoc} for information on the additional parameters of the {\tt adjustXXX} methods.

\subsubsection{Method {\tt updateDepencencies}}
This method describes the dependencies of the constraint on its variables.
It indicates to the solver when to filter the constraint 
(more precisely, call its {\tt filter} method) 
depending on events 
(more precisely, variable modifications).
\begin{verbatim}
   public void updateDependencies() {
      IntegerVariable x = (IntegerVariable)variables[0];
      IntegerVariable y = (IntegerVariable)variables[1];
      x.addMinConstraint(this);
      y.addMaxConstraint(this);
   }
\end{verbatim}
Here, we state that the constraint should be filtered whenever 
the minimal value of variable {\tt x} or 
the maximal value of variable {\tt y} changes.

\subsubsection{Complete code}}
The complete code of the $x \leq y$ constraint is eventually rather short and is given in
figure \ref{tab:code-leq}.
\begin{figure}[htb]
\begin{centeredframe}
\begin{verbatim}
public class Leq extends BinaryConstraint {
   public Leq(IntegerVariable x, IntegerVariable y) {
      super(x, y);
      name = "x<=y";
   }
   public void updateDependencies() {
      IntegerVariable x = (IntegerVariable)variables[0];
      IntegerVariable y = (IntegerVariable)variables[1];
      x.addMinConstraint(this);
      y.addMaxConstraint(this);
   }
   public void filter(ChoicePointStack cp, 
                      ConstraintScheduler cs) 
      throws ConstraintInconsistencyException {
      IntegerVariable x = (IntegerVariable)variables[0];
      IntegerVariable y = (IntegerVariable)variables[1];
      x.adjustMax(cp, cs, this, y.getMax());
      y.adjustMin(cp, cs, this, x.getMin());
   }
}
\end{verbatim}
\end{centeredframe}
\caption{Constraint $x \leq y$ source code\label{tab:code-leq}}
\end{figure}
Note that in this example, we have only used the bounds of the variables
domains. It is actually a very common case and justifies why there is an
implementation of domains with intervals. In more complex filtering
algorithms, one can take advantage of the explicit set of possible values, in
which case the domains have to be defined as {\tt SparseDomain}s.

%----------------------------------------------------------------------------
\section{Problems}
A problem is defined by 
a collection of constraints and a collection of constrained variables. 

\subsection{Basic use of problems}
Usually one defines a single
problem which is passed to the solver (as it was the case in section 
\ref{sec:send}):
\begin{verbatim}
Solver solver = new DefaultSplitSolver(pb);
\end{verbatim}

\subsection{Nested problems}
Let us consider the following problem:
\begin{verbatim}
public class Dist extends BaseProblem {
   public Dist(IntegerVariable d, 
               IntegerVariable x, 
               IntegerVariable y) {
      super();
      name = "d=|x-y|";
      IntegerVariable z = new IntegerVariable();
      add(new Add(x,y,z));
      add(new Abs(d,z));
      setVariables(new Variable[] {d,x,y});
   }
}
\end{verbatim}
This problem contains the {\tt Add} and {\tt Abs} constraints and the 
{\tt x}, {\tt y} and {\tt z} constrained variables. It also defines internally
the {\tt z} variable, which is the difference between {\tt x} and {\tt y}.
It is not difficult to see that this problem corresponds to the distance 
relation $d=|x-y|$.

Fortunately in {\jcs}, problems offer an interesting property: 
they can be added to a higher-level problem (eg the top-level problem) and thus 
can be re-used as constraints. 
Hence, we could write:
\begin{verbatim}
problem.add(new Dist(d, x, y));
\end{verbatim}
where {\tt problem} would just be another higher-level problem.

%----------------------------------------------------------------------------
\chapter{Solvers and heuristics}
\label{chap:solvers-heuristics}
This chapter goes into the details of solvers and 
explain how they use heuristics to solve the problems.

%----------------------------------------------------------------------------
\section{The hierarchy of solvers}
Solvers (control layer) are another pillar of {\jcs}: 
they control the resolution of the problem at hand.
This is the part that one is the most likely to customize, which is why
{\jcs} offers a comprehensive hierarchy of solvers, allowing precise implementation
or overriding of methods and thus fine-grained control over the solution search.
% MAT: tu as surement qque chose de precis en tete quand tu dis "allowing
% precise implementation or overriding of methods". Sans savoir de quoi il
% s'agit, je trouve que la phrase me laisse sur ma fin.
The most important elements of that hierarchy are described here. 

Refer to the \javadoc for a complete description of the hierarchy of solvers.

\subsection{General solvers}
% MAT: cela me semble inutile a ce stade
%Given a problem, a general solver is responsible for finding its solutions. 

In \jcs\, all solver implement the {\tt Solver} interface, which define their
common behavior.
Requesting a solver to solve its associated problem is achieved by:
\begin{verbatim}
solver.solve(n);
\end{verbatim}
where {\tt n} is the requested number of solutions. Requesting {\tt n}
solutions does not imply that the problem has more or less than {\tt n}
solutions. It means that once the solver has found {\tt n} solutions,
it will stop searching for additional ones. If there are less than {\tt n}
solutions to the problem, the solver will find all solutions and return after
proving that no other solution can be found.

When the solver is done, but before the {\tt solve} method returns, the {\tt
  solved} method is called.
By default\footnote{It can be overridden.} for all \jcs\ solvers, 
it displays the solutions and the solver counters.
The counters depend on the type of solver, 
but contain at least the total amount of time spent finding the solutions.

After the {\tt solve} method returns, 
the computed solutions can be accessed with the {\tt
  getSolutions} method, 
which returns a {\tt SolutionStack} (see the {\javadoc} for more details).

A useful \jcs\ implementation of the {\tt Solver} interface is the {\tt
  BaseSolver} class, an abstract class representing the
solvers that find the solutions one after another. This is the most 
common case of solution search\footnote{Problems with only linear
  equations as their constraints can, on the contrary, be solved globally.
  The corresponding solvers could, in such cases, directly implement the {\tt
  Solver} interface.}.

The {\tt BaseSolver} class has two interesting methods, that can both be
overriden:
\begin{itemize}
\item {\tt noMoreSolutions}, which is called when 
it is not possible to find the required number of solutions;
\item {\tt solutionFound}, which is called each time a new solution is found.
By default it displays the values of the counters relative to that solution.
\end{itemize}

\subsection{Solvers that know how to backtrack}
{\tt BacktrackSolver} is the abstract class of solvers that know how to
backtrack,  which means that they know how to explore choices.
Choices are decisions taken by the solver during the search for solutions,
just like the choices you would have to make at each crossing if you had to
get out of a labyrinth. They are represented in \jcs\ by the {\tt ChoicePoint}
class. 
% MAT: cette histoire de disjonction n'est pas intuitive pour le
% neophyte. Doit-on en parler ici?
% (they use {\tt ChoicePoint}s which implement disjunctions).

Backtrack solvers have additional counters that deal with 
the number of backtracks used to find the solutions.

{\tt SplitSolver} is a concrete sub-class of {\tt BacktrackSolver}.
It constructs choices by:
\begin{enumerate}
\item choosing a variable; 
\item choosing a new domain for this variable, for example by instantiating
  the variable.
\end{enumerate}
Let us consider a {\tt SplitSolver} instance and a variable $x$, chosen by the
solver, with $[1,4]$ as its initial domain. The choice to make for the domain
can mathematically be represented by the disjunction $x\in [1,2] \vee x \in
[3,4]$ (the domain is {\em split} in halves). If the solver decides to reduce
the domain of $x$ to $[1,2]$, it will memorize the domain $[3,4]$ in a {\tt
  ChoicePoint} instance and continue its search. If there is no solution with
$x \in [1,2]$, the solver will then backtrack to $x \in [3,4]$ and finish the
search in this domain.

The constructor of the {\tt SplitSolver} class takes three arguments:
\begin{itemize}
\item a problem;
\item a variable heuristic, used to select the variable;
\item a domain heuristic, used to create the new domain.
\end{itemize} 

Note that {\jcs} comes with the {\tt DefaultSplitSolver} class, which uses
standard heuristics and thus gets constructed with a single argument:
\begin{verbatim}
Solver solver = new DefaultSplitSolver(pb);
\end{verbatim}
which is equivalent to:
\begin{verbatim}
new SplitSolver(pb, 
                new KeepOrderVariableHeuristic(), 
                new IncreasingOrderDomainHeuristic());
\end{verbatim}

%----------------------------------------------------------------------------
\section{Heuristics}
We describe here the main heuristics included in {\jcs}.

\subsection{Variable heuristics}
A variable heuristic is responsible for selecting a variable among the problem
variables {\em which are not already instantiated}. As we have seen before,
such selections are needed by the solvers during the search for
solutions. Among others, the following variable heuristics are provided with
{\jcs}:
\begin{itemize}
\item {\tt KeepOrderVariableHeuristic} chooses the first variable;
\item {\tt SmallestVariableHeuristic} chooses the variable with the 
smallest domain (it corresponds to the first-fail principle);
\item {\tt MostConstrainedVariableHeuristic} chooses the variable involved in
  the highest number of constraints; % MAT: bien que je comprenne parfaitement
                                     % l'ancienne d�finition, je la trouve
                                     % finalement ambigue pour une premi�re
                                     % approche
\item {\tt MaxRegretVariableHeuristic} chooses the variable with the greatest
{\em regret}. It applies to cases when a cost is associated, for each
variable, to each value of the variable domain. For a given variable, the {\em
  regret} is then defined as the difference between its highest possible cost
and its second highest possible cost. The constructor of this class naturally
takes an argument defining the costs.
\end{itemize}
Refer to the {\javadoc} for the complete set of the variable heuristics.

\subsection{Domain heuristics}
A domain heuristic is responsible for creating a new sub-domain for the
selected variable, a critical operation during the search for solutions. Among
others, the following domain heuristics are provided with {\jcs}:
\begin{itemize}
\item {\tt IncreasingOrderDomainHeuristic} instantiates the variable with its
minimal possible value (the domain is reduced to a singleton);
\item {\tt SplitLowDomainHeuristic} splits the domain into halves and chooses
  the ``low'' half;
\item {\tt MinCostDomainHeuristic} instantiates the variable with a value 
of minimal cost (the domain is reduced to a singleton). The constructor of
this class naturally takes an argument defining the costs.
\end{itemize}
Refer to the {\javadoc} for the complete set of the domain heuristics.

%----------------------------------------------------------------------------
\chapter{Optimizers}
\label{chap:optimizers}
This chapter introduces the generic concept of optimizers. 
Optimizers are responsible for computing optimal solutions.

%----------------------------------------------------------------------------
\section{The hierarchy of optimizers}
Each optimizer being parameterized by a solver, 
there is a hierarchy of optimizers matching the solver hierarchy. 
Refer to the {\javadoc} for a complete description of the optimizer hierarchy.

The two concrete classes at the bottom of this hierarchy are: {\tt Minimizer}
and {\tt Maximizer}.
A {\tt Minimizer} instance is declared and defined as follows:
\begin{verbatim}
Optimizer minimizer = new Minimizer(varToMinimize,
                                    solver,        
                                    mode);
\end{verbatim}
The third constructor argument (the optimization mode) is described in the next
section.

{\tt Maximizer} instances are declared identically.

%----------------------------------------------------------------------------
\section{Optimization modes}
Optimizers find the global optimum by searching for a sequence of local
optima. 
They can work in two distinct modes,
{\tt Optimizer.RESTART} or {\tt Optimizer.CONTINUE}:
\begin{itemize}
\item in the {\tt RESTART} mode, each time a local optimum is found, the
  problem is enriched with a new constraint excluding that optimum from
  the solutions, and the entire search space is again explored "from the
  beginning" to search for the next local optimum;
\item in the {\tt CONTINUE} mode, the search space is explored incrementally.
\end{itemize}
The {\tt CONTINUE} mode generally gives better results, except for some
heuristics that require a well-balanced search space.

%----------------------------------------------------------------------------
\end{document}
