%----------------------------------------------------------------------------
\documentclass[a4paper]{article}
\usepackage{times}

%----------------------------------------------------------------------------
\input{../../../util/tex/products}
\def\version{3.1}


%----------------------------------------------------------------------------
\def\cp{Programmation par Contraintes}

%----------------------------------------------------------------------------
\begin{document}
\title{Pr\'esentation de {\jcs}}
\author{Copyright 2002 \koalog}
\date{}
\maketitle

\begin{center}
\begin{minipage}{9cm}
\begin{small}
{\sunlegal}
{\ibmlegal}
{\mslegal}
{\linuxlegal}
\end{small}
\end{minipage}
\end{center}

%----------------------------------------------------------------------------
\section{Introduction} % MAT: 'Overview' �a fait trop r�p�tition du titre.
{\jcs} est une librairie {\java} pour la r\'esolution de contraintes.
Avec cette librairie, il est possible de r\'esoudre des probl\`emes notoirement complexes tels que:
\begin{itemize}
\item l'ordonnancement;
\item l'allocation de ressources;
\item la planification;
\item la cr\'eation d'emploi du temps;
\item la configuration;
\item et plus g\'en\'eralement, tout probl\`eme de satisfaction ou d'optimisation.
\end{itemize}

{\jcs} inclut des r\'esolveurs sur les domaines entiers, bool\'eens et
ensemblistes, 
ainsi qu'une importante collection de contraintes pr\'ed\'efinies.

%----------------------------------------------------------------------------
\section{Contraintes incluses}
{\jcs} est livr\'e avec de nombreuses contraintes, r\'eparties dans les cat\'egories suivantes:
\begin{itemize}
\item contraintes arithm\'etiques (dont somme, diff\'erence, comparaison, valeur absolue);
\item contraintes bool\'eennes (dont conjonction, disjonction, n\'egation);
\item m\'eta-contraintes (dont tests de relations tels que: \'egalit\'e, comparaison);
\item contraintes globales (dont alldifferent, cumulative, cycle,
  disjunctive, global cardinality constraint, inverse, sort).
\end{itemize}
Chacune de ces contraintes, en particulier les contraintes globales, 
utilisent des algorithmes de filtrage \`a la pointe de la technologie.

%----------------------------------------------------------------------------
\section{Facilit\'e d'utilisation}
Habituellement, en {\cp}, 
les \'etapes \`a suivre pour la r\'esolution d'un probl\`eme sont: 
\begin{itemize}
\item la mod\'elisation du probl\`eme;
\item la conception d'un r\'esolveur adapt\'e au probl\`eme.
\end{itemize}
L'API {\java} de {\jcs} a \'et\'e con\c{c}ue de mani\`ere \`a ce que les utilisateurs n'aient pas besoin d'\^etre des sp\'ecialistes de la programmation par contraintes ni des experts {\java}:
\begin{itemize}
\item la mod\'elisation d'un probl\`eme s'effectue avec les classes fournies par {\jcs}, dont l'utilisation est intuitive;
\item les r\'esolveurs fournis avec {\jcs} sont efficaces et suffisamment flexibles pour convenir \`a la plupart des probl\`emes,
\'evitant le plus souvent la conception d'un r\'esolveur sp\'ecifique.
\end{itemize}

%----------------------------------------------------------------------------
\section{Architecture ouverte}

{\jcs} offre une architecture ouverte et extensible, \`a laquelle les utilisateurs exp\'eriment\'es peuvent facilement adjoindre leurs extensions. Le tableau \ref{tab:extensions} fournit de plus amples informations.
\begin{table}[htbp]
\begin{center}
\begin{tabular}{|p{4cm}|p{7cm}|}
\hline
{\bf Extension}                                                 & {\bf Temps
 de r\'ealisation} (d\'epend du niveau du d\'eveloppeur et de la complexit\'e)\\
\hline
\hline
Ajouter une nouvelle contrainte                                         & de quelques heures \`a deux ou trois jours\\
\hline
D\'evelopper une nouvelle strat\'egie de recherche                                & de quelques heures \`a 2 ou 3 jours\\
\hline
D\'evelopper un nouveau r\'esolveur (r\'esolveur flou, etc)                      & de 2 ou 3 jours \`a une quinzaine de jours\\
\hline
\end{tabular}
\caption{\label{tab:extensions}Exemples d'extensions}
\end{center}
\end{table}

%----------------------------------------------------------------------------
\section{Qualit\'e}
{\jcs} a \'et\'e r\'ealis\'e dans le respect de nombreux crit\`eres de qualit\'e:
\begin{itemize}
\item documentation compl\`ete (Javadoc et tutoriel);
\item internationalisable;
\item enti\`erement test\'e
(utilisation de {\junit}\footnote{{\junit} est un projet open-source cr\'e\'e par Kent Beck (auteur de ``Extreme Programming'').} 
et de {\koalog} Code Coverage pour les tests);
\item syst\`eme de logs enti\`erement param\'etrable ({\log4j}\footnote{{\log4j} est un projet open-source de l'{\asf}.}).
\end{itemize}

%----------------------------------------------------------------------------
\section{Applications}
\subsubsection*{Applications commerciales}
{\jcs} est actuellement utilis\'e par un configurateur destin\'e \`a l'industrie automobile.

\subsubsection*{Recherche}
Des exemples de code source montrant comment r\'esoudre des probl\`emes classiques tels que 
\begin{itemize}
\item la r\`egle de Golomb,
\item le voyageur de commerce,
\item l'ordonnancement d'ateliers,
\item les calendriers sportifs (dont les tournois {\em round-robin}),
\item les prob\`emes de partition,
\item le sac-\`a-dos,
\item et beaucoup d'autres
\end{itemize}
sont disponibles sur le site internet de {\jcs}:\\
{\tt http://www.koalog.com/php/jcs.php}.

%----------------------------------------------------------------------------
\section{Plates-formes support\'ees}
{\jcs} est une librairie {\java} et peut donc \^etre int\'egr\'ee ais\'ement \`a n'importe quelle application {\java}.
D\'etails dans le tableau \ref{tab:ports}.

\begin{table}[htbp]
\begin{center}
\begin{tabular}{|l||r|r|r|r|r|}
\hline
                  & \multicolumn{4}{|c|}{Machine Virtuelle {\java}} \\ \cline{2-5}
Syst\`eme d'exploitation  & \multicolumn{2}{|c|}{{\sun} {\jsdk}} & \multicolumn{2}{|c|}{{\ibm} JDK} \\ \cline{2-5}
                  & 1.3.1 & 1.4.2 & 1.3.1 & 1.4.1 \\
\hline
\hline
{\windowsXP}    & oui & oui & oui & n/a \\ 
{\suse} 9.1     & oui & oui & non & oui \\
{\fedoracore} 1 & oui & oui & non & oui \\
{\solaris} Sparc 2.8  & oui & oui & n/a & n/a \\
\hline
\end{tabular}
\caption{\label{tab:ports}Plate-formes support\'ees}
\end{center}
\end{table}

%----------------------------------------------------------------------------
\section{Informations commerciales}
D\'etails dans le tableau \ref{tab:commercial}.

\begin{table}[htbp]
\begin{center}
\begin{tabular}{|l||p{7cm}|}
\hline
Derni\`ere version & v\version \\
\hline
M\'ethodes de distribution & Ventes en direct \\
\hline
\end{tabular}
\caption{\label{tab:commercial}Informations commerciales}
\end{center}
\end{table}

%----------------------------------------------------------------------------
\end{document}

